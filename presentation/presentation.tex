%%%%%%%%%%%%%%%%%%%%%%%%%%%%%%%%%%%%%%%%%
% Beamer Presentation
% LaTeX Template
% Version 1.0 (10/11/12)
%
% This template has been downloaded from:
% http://www.LaTeXTemplates.com
%
% License:
% CC BY-NC-SA 3.0 (http://creativecommons.org/licenses/by-nc-sa/3.0/)
%
%%%%%%%%%%%%%%%%%%%%%%%%%%%%%%%%%%%%%%%%%

%----------------------------------------------------------------------------------------
%	PACKAGES AND THEMES
%----------------------------------------------------------------------------------------

\documentclass{beamer}

\mode<presentation> {

% The Beamer class comes with a number of default slide themes
% which change the colors and layouts of slides. Below this is a list
% of all the themes, uncomment each in turn to see what they look like.

%\usetheme{default}
%\usetheme{AnnArbor}
%\usetheme{Antibes}
%\usetheme{Bergen}
%\usetheme{Berkeley}
\usetheme{Berlin}
%\usetheme{Boadilla}
%\usetheme{CambridgeUS}
%\usetheme{Copenhagen}
%\usetheme{Darmstadt}
%\usetheme{Dresden}
%\usetheme{Frankfurt}
%\usetheme{Goettingen}
%\usetheme{Hannover}
%\usetheme{Ilmenau}
%\usetheme{JuanLesPins}
%\usetheme{Luebeck}
%\usetheme{Madrid}
%\usetheme{Malmoe}
%\usetheme{Marburg}
%\usetheme{Montpellier}
%\usetheme{PaloAlto}
%\usetheme{Pittsburgh}
%\usetheme{Rochester}
%\usetheme{Singapore}
%\usetheme{Szeged}
%\usetheme{Warsaw}

% As well as themes, the Beamer class has a number of color themes
% for any slide theme. Uncomment each of these in turn to see how it
% changes the colors of your current slide theme.

%\usecolortheme{albatross}
%\usecolortheme{beaver}
%\usecolortheme{beetle}
%\usecolortheme{crane}
%\usecolortheme{dolphin}
%\usecolortheme{dove}
%\usecolortheme{fly}
%\usecolortheme{lily}
%\usecolortheme{orchid}
%\usecolortheme{rose}
%\usecolortheme{seagull}
\usecolortheme{seahorse}
%\usecolortheme{whale}
%\usecolortheme{wolverine}

%\setbeamertemplate{footline} % To remove the footer line in all slides uncomment this line
%\setbeamertemplate{footline}[page number] % To replace the footer line in all slides with a simple slide count uncomment this line
\setbeamertemplate{headline}{}
\setbeamertemplate{navigation symbols}{} % To remove the navigation symbols from the bottom of all slides uncomment this line
}

  \setbeamertemplate{footline}%{miniframes theme}
  {%
    \begin{beamercolorbox}[colsep=1.5pt]{upper separation line foot}
    \end{beamercolorbox}
    \begin{beamercolorbox}[ht=2.5ex,dp=1.125ex,%
      leftskip=.3cm,rightskip=.3cm plus1fil]{author in head/foot}%
      \leavevmode{\usebeamerfont{author in head/foot}\insertshortauthor}%
      \hfill%
      {\usebeamerfont{institute in head/foot}\usebeamercolor[fg]{institute in head/foot}\insertshortinstitute}%
    \end{beamercolorbox}%
    \begin{beamercolorbox}[ht=2.5ex,dp=1.125ex,%
      leftskip=.3cm,rightskip=.3cm plus1fil]{title in head/foot}%
      {\usebeamerfont{title in head/foot}\insertshorttitle} \hfill     \insertframenumber/\inserttotalframenumber%
    \end{beamercolorbox}%
    \begin{beamercolorbox}[colsep=1.5pt]{lower separation line foot}
    \end{beamercolorbox}
  }


\usepackage{listings}
\usepackage{graphicx} % Allows including images
\usepackage{booktabs} % Allows the use of \toprule, \midrule and \bottomrule in tables
\usepackage[utf8]{inputenc}
\usepackage[magyar]{babel}
\usepackage{fixltx2e}
\usepackage{epstopdf}
\usepackage{xcolor}



%----------------------------------------------------------------------------------------
%	TITLE PAGE
%----------------------------------------------------------------------------------------

\title[PredictLikes]{ DumboAsAService \\ PredictLikes: Blog like-profil klaszterezés} % The short title appears at the bottom of every slide, the full title is only on the title page

\author[DumboAsAService]{Nótai István (notaiistv@gmail.com)\\ Szakállas Dávid(david.szakallas@gmail.com)	\\ Mátyás-Barta Csongor(mbcsongor@gmail.com)} % Your name

\institute[BME]{} % Your institution as it will appear on the bottom of every slide, may be shorthand to save space

\date{\today} % Date, can be changed to a custom date

\begin{document}


\begin{frame}[plain]
\titlepage % Print the title page as the first slide
\end{frame}

%----------------------------------------------------------------------------------------
%	PRESENTATION SLIDES
%----------------------------------------------------------------------------------------
\begin{frame}[fragile]
\frametitle{Feladat értelmezés}
PredictLikes - blogok "like-profil" alapján klaszterezése és a csoportosítás vizualizációja:

Like profil  a mi értelmezésünkben = Ha két blognak van közös likeolója, akkor ők hasonló blogok.
\\~\\
\begin{tabular}{cl}  
         \begin{tabular}{c}
		 
			 \parbox{0.5\linewidth}{%
			\{"uid": "34168956",   \\
			"likes": 
			
			[\{"blog": "18164949"\},  
			
			\{"blog": "6789"\}]\}
			}
           \end{tabular}
           & \begin{tabular}{c}
              \includegraphics[scale=0.6,trim = 3.5cm 16.5cm 10cm 9cm, clip]{figures/graf.pdf}
         \end{tabular}  \\
\end{tabular}

\end{frame}

%----------------------------------------------------------------------------------------
\begin{frame}
\frametitle{MapReduce - Éllista építés}
\includegraphics[scale=0.6,trim = 2cm 17cm 3cm 0cm, clip]{figures/mapreduce.pdf} \\
Reduce: csak a 10-nél nagyobb súlyú éleket írjuk ki
\end{frame}

%----------------------------------------------------------------------------------------
\begin{frame}
\frametitle{Community Search} 
Fastgreedy community search [1]
\begin{itemize}
\item bottom-up hierarchikus megközelítés
\item kezdetben minden csomópont egy közösség
\item greedy: melyik két közösség összevonása eredményezné a legnagyobb \emph{modularitás} növekedést lokálisan
\item rezolúciós hiányosság: a gráf egészéhez képest relatív kis közösségek mindig összevonódnak
\end{itemize}
\end{frame}

%----------------------------------------------------------------------------------------
\begin{frame}
\frametitle{Végeredmény és vizualizáció}
Közel 7000 csomópont és 700k él  $\rightarrow$ 54 közösség\\~\\
\begin{columns}[t]
	
	\column{.5\textwidth} % Left column and width
		\centering
		igraph \\
		\includegraphics[width = 5cm, trim = 7.5cm 1cm 5.5cm 1cm, clip]{figures/groups.pdf}
	\column{.5\textwidth} % Right column and width
		\centering
		BioFabric \\~\\
		\includegraphics[width = 5cm, trim = 2cm 3.5cm 1cm 1.5cm, clip]{figures/biofabric.pdf}
\end{columns}
\end{frame}

%----------------------------------------------------------------------------------------
\begin{frame}
\frametitle{Használt eszközök} % Table of contents slide, comment this block out to remove it
\begin{columns}[t]
	\column{.5\textwidth} % Left column and width
		Éllista építés
      \begin{itemize}
        \item local Hadoop MapReduce(Java)
        \item 15GB RAM
      \end{itemize}
	\column{.5\textwidth} % Right column and width
		Gráf feldolgozás
      \begin{itemize}
		\item R
        \item igraph R csomag [2]
		\item BioFabric R port [3]
      \end{itemize}
\end{columns}
\end{frame}


%------------------------------------------------
\begin{frame}[plain]
\Huge{\centerline{Köszönjük a figyelmet!}}
\end{frame}

%------------------------------------------------
\begin{frame}[plain]
\frametitle{Hivatkozások}
\small{
\begin{enumerate}
\item http://arxiv.org/abs/cond-mat/0408187
\item http://igraph.org/r/
\item https://github.com/wjrl/RBioFabric/blob/master/R/bioFabric.R
\end{enumerate}
}
\end{frame}


%------------------------------------------------
\end{document}
